%% Generated by Sphinx.
\def\sphinxdocclass{report}
\documentclass[letterpaper,10pt,english]{sphinxmanual}
\ifdefined\pdfpxdimen
   \let\sphinxpxdimen\pdfpxdimen\else\newdimen\sphinxpxdimen
\fi \sphinxpxdimen=.75bp\relax

\PassOptionsToPackage{warn}{textcomp}
\usepackage[utf8]{inputenc}
\ifdefined\DeclareUnicodeCharacter
 \ifdefined\DeclareUnicodeCharacterAsOptional
  \DeclareUnicodeCharacter{"00A0}{\nobreakspace}
  \DeclareUnicodeCharacter{"2500}{\sphinxunichar{2500}}
  \DeclareUnicodeCharacter{"2502}{\sphinxunichar{2502}}
  \DeclareUnicodeCharacter{"2514}{\sphinxunichar{2514}}
  \DeclareUnicodeCharacter{"251C}{\sphinxunichar{251C}}
  \DeclareUnicodeCharacter{"2572}{\textbackslash}
 \else
  \DeclareUnicodeCharacter{00A0}{\nobreakspace}
  \DeclareUnicodeCharacter{2500}{\sphinxunichar{2500}}
  \DeclareUnicodeCharacter{2502}{\sphinxunichar{2502}}
  \DeclareUnicodeCharacter{2514}{\sphinxunichar{2514}}
  \DeclareUnicodeCharacter{251C}{\sphinxunichar{251C}}
  \DeclareUnicodeCharacter{2572}{\textbackslash}
 \fi
\fi
\usepackage{cmap}
\usepackage[T1]{fontenc}
\usepackage{amsmath,amssymb,amstext}
\usepackage{babel}
\usepackage{times}
\usepackage[Bjarne]{fncychap}
\usepackage{sphinx}

\usepackage{geometry}

% Include hyperref last.
\usepackage{hyperref}
% Fix anchor placement for figures with captions.
\usepackage{hypcap}% it must be loaded after hyperref.
% Set up styles of URL: it should be placed after hyperref.
\urlstyle{same}

\addto\captionsenglish{\renewcommand{\figurename}{Fig.}}
\addto\captionsenglish{\renewcommand{\tablename}{Table}}
\addto\captionsenglish{\renewcommand{\literalblockname}{Listing}}

\addto\captionsenglish{\renewcommand{\literalblockcontinuedname}{continued from previous page}}
\addto\captionsenglish{\renewcommand{\literalblockcontinuesname}{continues on next page}}

\addto\extrasenglish{\def\pageautorefname{page}}

\setcounter{tocdepth}{5}
\setcounter{secnumdepth}{5}


\title{tectTutorial Documentation}
\date{Apr 11, 2020}
\release{Apr-202}
\author{Prashant}
\newcommand{\sphinxlogo}{\vbox{}}
\renewcommand{\releasename}{Release}
\makeindex

\begin{document}

\maketitle
\sphinxtableofcontents
\phantomsection\label{\detokenize{index::doc}}



\chapter{Installation}
\label{\detokenize{install:installation}}\label{\detokenize{install::doc}}\label{\detokenize{install:welcome-to-tecttutorial-s-documentation}}
At the command line:

\fvset{hllines={, ,}}%
\begin{sphinxVerbatim}[commandchars=\\\{\}]
\PYG{n}{easy\PYGZus{}install} \PYG{n}{crawler}
\end{sphinxVerbatim}

Or, if you have virtualenvwrapper installed:

\fvset{hllines={, ,}}%
\begin{sphinxVerbatim}[commandchars=\\\{\}]
mkvirtualenv crawler
pip install crawler
\end{sphinxVerbatim}


\chapter{Support}
\label{\detokenize{source:support}}\label{\detokenize{source::doc}}
The easiest way to get help with the project is to join the \sphinxcode{\sphinxupquote{\#crawler}}
channel on \sphinxhref{irc://freenode.net}{Freenode}. We hang out there and you can get real-time help with
your projects.  The other good way is to open an issue on \sphinxhref{http://github.com/example/crawler/issues}{Github}.

The mailing list at \sphinxurl{https://groups.google.com/forum/\#!forum/crawler} is also available for support.


\chapter{Cookbook}
\label{\detokenize{cookbook:github}}\label{\detokenize{cookbook::doc}}\label{\detokenize{cookbook:cookbook}}

\section{Crawl a web page}
\label{\detokenize{cookbook:crawl-a-web-page}}
The most simple way to use our program is with no arguments.
Simply run:

\fvset{hllines={, ,}}%
\begin{sphinxVerbatim}[commandchars=\\\{\}]
\PYG{n}{crawler} \PYG{o}{\PYGZlt{}}\PYG{n}{url}\PYG{o}{\PYGZgt{}}
\end{sphinxVerbatim}

to crawl a webpage.


\section{Crawl a page slowly}
\label{\detokenize{cookbook:crawl-a-page-slowly}}
To add a delay to your crawler,
use \sphinxhref{https://docs.python.org/3/using/cmdline.html\#cmdoption-d}{\sphinxcode{\sphinxupquote{-d}}}:

\fvset{hllines={, ,}}%
\begin{sphinxVerbatim}[commandchars=\\\{\}]
\PYG{n}{crawler} \PYG{o}{\PYGZhy{}}\PYG{n}{d} \PYG{l+m+mi}{10} \PYG{o}{\PYGZlt{}}\PYG{n}{url}\PYG{o}{\PYGZgt{}}
\end{sphinxVerbatim}

This will wait 10 seconds between page fetches.


\section{Crawl only your blog}
\label{\detokenize{cookbook:crawl-only-your-blog}}
You will want to use the \sphinxhref{https://docs.python.org/3/using/cmdline.html\#cmdoption-i}{\sphinxcode{\sphinxupquote{-i}}} flag,
which while ignore URLs matching the passed regex:

\fvset{hllines={, ,}}%
\begin{sphinxVerbatim}[commandchars=\\\{\}]
\PYG{n}{crawler} \PYG{o}{\PYGZhy{}}\PYG{n}{i} \PYG{l+s+s2}{\PYGZdq{}}\PYG{l+s+s2}{\PYGZca{}blog/}\PYG{l+s+s2}{\PYGZdq{}} \PYG{o}{\PYGZlt{}}\PYG{n}{url}\PYG{o}{\PYGZgt{}}
\end{sphinxVerbatim}

This will only crawl pages that contain your blog URL.
\begin{quote}
\begin{quote}

You will want to use the \sphinxhref{https://docs.python.org/3/using/cmdline.html\#cmdoption-i}{\sphinxcode{\sphinxupquote{-i}}} flag,
which while ignore URLs matching the passed regex:

\fvset{hllines={, ,}}%
\begin{sphinxVerbatim}[commandchars=\\\{\}]
\PYG{n}{crawler} \PYG{o}{\PYGZhy{}}\PYG{n}{i} \PYG{l+s+s2}{\PYGZdq{}}\PYG{l+s+s2}{pdf\PYGZdl{}}\PYG{l+s+s2}{\PYGZdq{}} \PYG{o}{\PYGZlt{}}\PYG{n}{url}\PYG{o}{\PYGZgt{}}
\end{sphinxVerbatim}

This will ignore URLs that end in PDF.
\end{quote}
\end{quote}


\chapter{Indices and tables}
\label{\detokenize{index:indices-and-tables}}\begin{itemize}
\item {} 
\DUrole{xref,std,std-ref}{genindex}

\item {} 
\DUrole{xref,std,std-ref}{modindex}

\item {} 
\DUrole{xref,std,std-ref}{search}

\end{itemize}



\renewcommand{\indexname}{Index}
\printindex
\end{document}